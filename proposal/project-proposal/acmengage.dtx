%
% ACM Engage course material
%
%<*acmengage>
%%
%% 
%% Commands for TeXCount
%<<TCMACROS
%TC:macro \cite [option:text,text]
%TC:macro \citep [option:text,text]
%TC:macro \citet [option:text,text]
%TC:envir table 0 1
%TC:envir table* 0 1
%TC:envir tabular [ignore] word
%TC:envir displaymath 0 word
%TC:envir math 0 word
%TC:envir comment 0 0
%TCMACROS
%% 
%% 
%% The first command in your LaTeX source must be the \documentclass command.
\documentclass[acmengage]{acmart}
\settopmatter{printacmref=false, printccs=false, printfolios=true}
\setcopyright{none}
\renewcommand\footnotetextcopyrightpermission[1]{}
\renewcommand{\shortauthors}{}
\renewcommand\acmBooktitle{}
\pagestyle{plain}


%% \BibTeX command to typeset BibTeX logo in the docs
\AtBeginDocument{%
  \providecommand\BibTeX{{%
    Bib\TeX}}}

%% Rights management information.  This information is sent to you
%% when you complete the rights form.  These commands have SAMPLE
%% values in them; it is your responsibility as an author to replace
%% the commands and values with those provided to you when you
%% complete the rights form.  Note that by default course materials
%% use Creative Commons license

% \setcopyright{cc}
% \setcctype{by}
% \copyrightyear{2022}
% \acmYear{May 2022}
% \acmBooktitle{none}
% \acmDOI{XXXXXXX.XXXXXXX}

\begin{document}

\fancyhead{} % Clear all headers
\fancyfoot{} % Clear all footers  
\fancyfoot[C]{\thepage} % Keep page numbers centered at bottom


\title{Graphing the Stars: Graph Queries in SiriusDB}
\author{Atharv Kirtikar}
\email{aakirtikar@cs.wisc.edu}
\affiliation{%
  \institution{University of Wisconsin-Madison}}

\author{Elliott Weinshenker}
\email{eweinshenker@cs.wisc.edu}
\affiliation{%
  \institution{University of Wisconsin-Madison}}


\author{Yayen Lin}
\email{lin383@wisc.edu}
\affiliation{%
  \institution{University of Wisconsin-Madison}}

%% The synopsis is a name for the abstract
\begin{abstract}
SiriusDB has demonstrated strong performance in analytical workloads through GPU acceleration, yet it currently lacks support for spatial and graph operations that are becoming more critical for modern data-intensive applications. This project aims to extend Sirius with graph capabilities for native GPU execution. Rather than reimplementing spatial functionality from scratch, we will leverage DuckDB community extensions for query parsing, type handling, and logical planning. Our approach maintains Sirius as a composable execution engine, either integrating existing GPU-native spatial libraries or developing custom GPU kernels for critical operations. 
\end{abstract}

%% Metadata for the course
% \setengagemetadata{Course}{CS764}
% \setengagemetadata{Programming Language}{Python}
% \setengagemetadata{Knowledge Unit}{Programming Concepts}
% \setengagemetadata{CS Topics}{Functions, Data Types, Expressions,
%   Mathematical Reasoning}

%% Keywords
\keywords{SiriusDB, OpenGIS, GQL Operators, OpenGIS, ST\_Spatial, ST\_GeomFromText, DuckDB}
\maketitle

\section{Introduction}
SiriusDB has demonstrated exceptional performance in analytical workloads by leveraging GPU acceleration for high-throughput query processing and vectorized computation. However, many emerging data-intensive applications require understanding relationships between entities—an area more naturally expressed and computed through graph models. This proposal introduces native graph query capabilities within SiriusDB, integrating them directly into its existing GPU execution framework.

By implementing graph primitive operators to support algorithms such as shortest distance, we aim to extend these primitives toward more complex graph algorithms including shortest path and connected components. This approach unifies relational and graph processing under a single, GPU-optimized engine, eliminating the need to fall back to host databases like DuckDB for graph queries. It will further exploit GPU parallelism for graph traversal, enabling SiriusDB to provide standardized support for both relational and graph analytics.

As a drop-in accelerator, SiriusDB offers GPU residency for a wide range of analytical and relational query operations. Its integration with Substrait facilitates translation between logical query representations and physical GPU execution, leveraging CPU-based query parsing and plan generation. Currently, SiriusDB supports DuckDB and Doris as host databases—these act as fallback execution engines when GPU features are unavailable\cite{Yogatama26}. While this design reduces complexity, it introduces friction in data movement and coordination between systems. Additionally, distributed query execution relies on the host database’s coordinator, which can become a performance bottleneck.

\section{Methodology}
Our goal is to extend SiriusDB’s functionality with OpenGIS-compliant operators implemented natively on the GPU. Specifically, we aim to support the \texttt{ST\_Distance} \cite{DuckPGQ} and \texttt{ST\_Ge\-om\-From\-Text} \cite{DuckDBSpatial} functions, currently implemented in the DuckDB ecosystem through the DuckPGQ and DuckDB Spatial extensions. These community-driven projects have already established the groundwork for integrating the GEOMETRY type and parsing graph-based data representations. To maintain SiriusDB’s role as a composable execution engine, we will leverage their existing work to generate Substrait logical query plans. Where operators are unsupported, we will evaluate whether the required primitives can be mapped to existing GPU-native operators or, where necessary, develop custom kernels to extend SiriusDB’s graph processing capabilities.
\section{Timeline}

Table~\ref{tab:timeline} provides an overview of our project development timeline.

\begin{table}[h!]
\caption{Project Timeline}
\label{tab:timeline}
\begin{tabular}{ccl}
\toprule
Phase & Date & Goal\\
\midrule
1 & Weeks 1--3 & System Setup and Preparation\\
2 & Weeks 4--7 & Implementation and Integration\\
3 & Weeks 8--10 & Evaluation and Documentation\\
\bottomrule
\end{tabular}
\end{table}

\subsection{Phase 1: System Setup and Preparation}
During this phase (Oct 17--Nov 2), we will explore the Sirius and DuckDB codebases, focusing on the operators used by DuckDB and how those translate to supported operators in Sirius. We will test several graph workloads with the aforementioned DuckDB extensions and configure access to GPU resources for Sirius.

\subsection{Phase 2: Implementation and Integration}
During this phase (Nov 8--Dec 5), we will port the core spatial functions \texttt{ST\_Distance}, \texttt{ST\_Ge\-om\-From\-Text} to libcudf geometry primitives. We will integrate these spatial functions into the Sirius operator pipeline and test end-to-end execution (DuckDB spatial query $\rightarrow$ Substrait plan $\rightarrow$ Sirius GPU execution). Additionally, we will examine DuckPGQ's Substrait plan output for simple graph queries (e.g., shortest path) and identify specific operators or patterns that Sirius cannot currently handle.

\subsection{Phase 3: Evaluation and Documentation}
During this phase (Dec 6--Dec 14), we will conduct performance evaluation, analyze our findings, and complete final documentation.  We will assess the feasibility of full graph query GPU execution, identify future work and remaining technical challenges, and write a comprehensive project report with methodology, results, and analysis.

% \section{Meta-Data}

% This section is included in the template to explain the choices for the meta-data at the top of the paper. It should not be included in the final paper submission.

% \subsection{Course}

% Current courses are:

% \begin{itemize}
%     \item CS0---a breadth first introductory computing course similar to Exploring Computer Science or AP CS Principles
%     \item CS1---an introductory programming course covering topics normally associated with an imperative or functional programming course. Similar to an AP CS A course
%     \item Data Structures---a follow-on course occurring after CS1 that introduces linear and non-linear data structures including implementation and usage
%     \item Discrete Math---a course covering discrete mathematical
%       structures such as integers, graphs and logic statements. This
%       may include logic, set theory, combinatorics, graphy theory,
%       number theory, topology, etc.
%     \item HCI---a course in the general area of human computer
%       interaction. This might be a general HCI course or a course in a
%       specific subdiscipline such as user-centred design.
% \end{itemize}

% More than one course may be selected. If you are submitting an OER for a special topics issue of Engage, please discuss the appropriate course choice with the guest editors of the special issue.

% \subsection{Programming Language}
% Authors may select all that apply from the following list:
% \begin{itemize}
%     \item C
%     \item C++
%     \item C\#
%     \item Java
%     \item JavaScript
%     \item Processing
%     \item Python
%     \item Racket (DrScheme)
%     \item Scheme
%     \item Scratch
%     \item Pseudocode
%     \item Other
%     \item None
% \end{itemize}

% \subsection{Resource Type}
% One resource type must be selected. Current list to select from includes:

% \begin{itemize}
%     \item Assignment---the most common OER type. Typically represents a task assigned to individual or groups of students that will be completed outside of class time.
%     \item Lecture slides---an annotated set of presentation slides to introduce or explain a topic, typically a cutting-edge research topic, to a more lay audience. An example might be explaining a specific cryptography algorithm, blockchain, or an AI / ML solution to a problem.
%     \item Lab---this represents a task assigned to an individual or group of students to be completed under supervision, usually during a closed-lab model
%     \item Project---an assignment that is of a longer duration, perhaps multiple weeks to an entire term
%     \item Tutorial---a task usually completed by an individual to learn some material on their own
%     \item Other---any other type of OER that does not fit into one of the above categories
% \end{itemize}

% \subsection{CS Concepts}
% This is selectable from the ontology of topics found at \url{https://www.engage-csedu.org/ontology}. Up to three topics may be selected. Eventually this page will be a tool allowing you to select up to three nodes in the tree and then copy / paste the descriptive text into your document and the submission system.

% \subsection{Knowledge Unit}
% Authors will select the most appropriate one from the following list:

% \begin{itemize}
%     \item Programming Concepts---anything involving programming
%     \item Data Structures---anything involving data structures
%     \item Software Development Methods---if the OER centers around software development (i.e., requirements gathering, testing, maintenance, code reviews) rather than the actual programming content
%     \item Discrete Math---anything involving discrete math
%     \item N/A---not applicable
% \end{itemize}

% % \subsection{Creative Commons License}
% During the submission process on ScholarOne, authors will select one create commons license from the following list:

% \begin{itemize}
%     \item CC BY-SA
%     \item CC BY-NC
%     \item CC BY-NC-ND
%     \item CC BY-NC-SA
%     \item CC BY-ND
%     \item CC BY
% \end{itemize}

% The correct typesetting of materials under creative commons license
% requires the corresponding CC icon. A modern \TeX\ distribution
% includes these icons in the package \textsl{doclicense}
% \cite{doclicense}.  In case your distribution does not have them, ACM
% provides a file \path{ccicons.zip} with these icons.  Just unzip it in
% the same directory where your document is.

% More information on Creative Common Licensing may be found at \url{https://creativecommons.org/licenses/}.

% \section{Submission}
% When you make a submission using ScholarOne you must upload:

% \begin{itemize}
%     \item an anonymized version of this paper for review
%     \item a zipped file containing all the student-facing materials. The materials in this file must also be anonymized for the purposes of fully anonymous review.
% \end{itemize}

% \section{Citations and References}
% \label{sec:citations}
% We recommend using \BibTeX\ to prepare your references. The bibliography is included 
% in your source document with these two commands, placed just before 
% the \verb|\end{document}| command:
% \begin{verbatim}
%   \bibliographystyle{ACM-Reference-Format}
%   \bibliography{bibfile}
% \end{verbatim}
% where ``\verb|bibfile|'' is the name, without the ``\verb|.bib|''
% suffix, of the \BibTeX\ file.

% Here are a few examples of the types of things you might cite in an EngageCSEdu submission: 
%   a book \cite{Kosiur01},
%   a journal article \cite{Abril07}, 
%   an informally published work \cite{Harel78}, 
%   an online document / world wide web resource \cite{Thornburg01, Ablamowicz07}, 
%   a video \cite{Obama08}, 
%   a software package \cite{R}, and an online dataset \cite{UMassCitations}.

% For other examples, see the file sample-acmsmall-conf.tex \cite{CTANacmart}.

% \section{Auxiliary Materials}
% This section is optional, but if included must immediately precede the References section. If there are no References, Auxiliary Materials should be last. This should be
% a numbered list of URLs with an optional brief description of the content found at each URL. Here is an example.
% \begin{enumerate}
% \item \url{https://somenews.org/xxx/}  A news article relevant to this OER.
% \item \url{https://somesite.gov/xxx/} A relevant government report.
% \item \url{https://someplace.edu/xxxx/} A public data set of interest. 
% \item \url{https://github.com/xxxx/} A public github project that is related.
% \end{enumerate} 

%%
%% The next two lines define the bibliography style to be used, and
%% the bibliography file.
\bibliographystyle{ACM-Reference-Format}
\bibliography{base}


\end{document}
%</acmengage>
